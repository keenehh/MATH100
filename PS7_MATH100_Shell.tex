\documentclass[11pt]{article}
%------------------------
%Packages
\usepackage[top=0.75in, bottom=1.25in, left=1in, right=1in]{geometry} 
\usepackage{amsmath,amsthm,amssymb} %this is THE math package
\usepackage{mathtools}
\usepackage{tikz}
\usepackage{graphicx}
\usepackage{enumitem}
\usepackage{fancybox}
\usepackage{hyperref}
\usepackage{varwidth}
\usepackage{mdframed}
\usepackage{mathrsfs}
%------------------------
%Fonts I use, uncomment if you like to use them.
%The first is the general font, and the second a math font
\usepackage{mathpazo}
%\usepackage{eulervm}
%------------------------
%This is so that we have standard fonts for the doublestroked symbols
%for reals, naturals etc. regardless of what font you use.
%Don't comment
\AtBeginDocument{
  \DeclareSymbolFont{AMSb}{U}{msb}{m}{n}
  \DeclareSymbolFontAlphabet{\mathbb}{AMSb}}

%----------------------------------------------
%User-defined environments
%Commented because we're not using them in this document
%The only uncommented ones are the Problem and Solution environment

% \newenvironment{theorem}[2][Theorem]{\begin{trivlist}
% \item[\hskip \labelsep {\bfseries #1}\hskip \labelsep {\bfseries #2.}]}{\end{trivlist}}
% \newenvironment{lemma}[2][Lemma]{\begin{trivlist}
% \item[\hskip \labelsep {\bfseries #1}\hskip \labelsep {\bfseries #2.}]}{\end{trivlist}}
% \newenvironment{exercise}[2][Exercise]{\begin{trivlist}
% \item[\hskip \labelsep {\bfseries #1}\hskip \labelsep {\bfseries #2.}]}{\end{trivlist}}
% \newenvironment{question}[2][Question]{\begin{trivlist}
% \item[\hskip \labelsep {\bfseries #1}\hskip \labelsep {\bfseries #2.}]}{\end{trivlist}}
% \newenvironment{corollary}[2][Corollary]{\begin{trivlist}
% \item[\hskip \labelsep {\bfseries #1}\hskip \labelsep {\bfseries #2.}]}{\end{trivlist}}
\newenvironment{problem}[2][Problem\!]{\begin{trivlist}
\item[\hskip \labelsep {\bfseries #1}\hskip \labelsep {\bfseries #2.}]}{\end{trivlist}}
%\newenvironment{sub-problem}[2][]{\begin{trivlist}
%\item[\hskip \labelsep {\bfseries #1}\hskip \labelsep {\bfseries #2}]}{\end{trivlist}}
\newenvironment{solution}{\begin{proof}[\textbf{\textit{Solution}}]}{\end{proof}}
%----------------------------------------------

%----------------------------
%User-defined notations
\newcommand{\zz}{\mathbb Z}   %blackboard bold Z
\newcommand{\qq}{\mathbb Q}   %blackboard bold Q
\newcommand{\ff}{\mathbb F}   %blackboard bold F
\newcommand{\rr}{\mathbb R}   %blackboard bold R
\newcommand{\nn}{\mathbb N}   %blackboard bold N
\newcommand{\cc}{\mathbb C}   %blackboard bold C
\newcommand{\af}{\mathbb A}   %blackboard bold A
\newcommand{\pp}{\mathbb P}   %blackboard bold P
\newcommand{\id}{\operatorname{id}} %for identity map
\newcommand{\im}{\operatorname{im}} %for image of a function
\newcommand{\dom}{\operatorname{dom}} %for domain of a function
\newcommand{\cat}[1]{\mathscr{#1}}   %calligraphic category
\newcommand{\abs}[1]{\left\lvert#1\right\rvert} %for absolute value
\newcommand{\norm}[1]{\left\lVert#1\right\rVert} %for norm
\newcommand{\modar}[1]{\text{ mod }{#1}} %for modular arithmetic
\newcommand{\set}[1]{\left\{#1\right\}} %for set
\newcommand{\setp}[2]{\left\{#1\ \middle|\ #2\right\}} %for set with a property
\newcommand{\card}[1]{\#\,{#1}} %for cardinality of a set

%Re-defined notations
\renewcommand{\epsilon}{\varepsilon}
\renewcommand{\phi}{\varphi}
\renewcommand{\emptyset}{\varnothing}
\renewcommand{\geq}{\geqslant}
\renewcommand{\leq}{\leqslant}
\renewcommand{\Re}{\operatorname{Re}}
\renewcommand{\gcd}{\operatorname{GCD}}
\renewcommand{\Im}{\operatorname{Im}}
%----------------------------

\allowdisplaybreaks
 
\begin{document}
 
\title{Problem Set 7}
\author{[Keene Ho]\\[0.5em]
MATH 100 | Introduction to Proof and Problem Solving | Summer 2023}
\date{} 
\maketitle

%Use \[...\] instead of $$...$$

\begin{problem}{7.1}
Let $A = \set{1, 2, 3, 4}$. Give an example, with reasoning, of a relation on $A$ that is:
\begin{itemize}[itemsep=2em]
\item[(a)] reflexive and symmetric but not transitive.
%----------------------------------------
\begin{solution}
We can define a relation \(R\) on \(A\) that follows:\\
\[R = \set{(1, 1), (2, 2), (3, 3), (4, 4), (1, 2), (2, 1), (3, 4), (4, 3), (1, 3)}\]
We can see that \((1,1),(2,2)(3,3),(4,4) \in R\) which makes it reflexive. We can also see that \((1,2),(2,1) \in R\) and \((3,4),(4,3) \in R\) which makes it symmetric. This is not transitive because we see that \((1,2),(4,4) \in R\) but not \((1,4) \notin R\).

\end{solution}
%----------------------------------------

\item[(b)] reflexive and transitive but not symmetric.
%----------------------------------------
\begin{solution}
%Uncomment and WRITE YOUR SOLUTION HERE
We can define a relation \(R\) on \(A\) that follows:
\[R = \set{(1,1),(2,2),(3,3),(4,4),(2,3),(1,3),(1,2)}\]
Here we see that \(R\) is reflexive and transitive but not symmetric. We can see that \((1,1),(2,2)(3,3),(4,4) \in R\) which makes it reflexive. For transitive we can see that \((2,3),(1,2) \in R\)
and \((2,2) \in R\) as well. This is not symmetric because we can see that \(1,3 \in R\) but not \(3,1 \notin R\).
\end{solution}
%----------------------------------------

\item[(c)] symmetric and transitive but not reflexive.
%----------------------------------------
\begin{solution}
%Uncomment and WRITE YOUR SOLUTION HERE
We can define a relation \(R\) on \(A\) that follows:
\[R = \set{(1,2),(2,1),(2,3),(3,2),(3,4),(4,3),(1,3),(2,3),(3,1),(3,3),(2,4)}\]
Here we see that \(R\) isn't symmetric because we don't have \((1,1),(2,2),(4,4) \notin R\). We see that \((2,3),(3,4) \in R\) then \((2,4) \in R\) which makes it transitive. We also see that its symmetric because \((1,2),(2,1) \in R\), \((2,3),(3,2) \in R\), and \((3,4),(4,3) \in R\).
\end{solution}
%----------------------------------------

\item[(d)] reflexive but neither symmetric nor transitive.
%----------------------------------------
\begin{solution}
%Uncomment and WRITE YOUR SOLUTION HERE
We can define a relation \(R\) on \(A\) that follows:
\[R = \set{(1,1),(2,2),(3,3),(4,4), (2,4), (4,3)}\]
Here we see that \(R\) is reflexive because \((1,1),(2,2)(3,3),(4,4) \in R\). It is not symmetric because \((2,4) \in R\) but not \((4,2) \notin R\). It is also not transitive because \((2,4),(4,3) \in R\) but not \((2,3) \notin R\).
\end{solution}
%----------------------------------------

\item[(e)] symmetric but neither reflexive nor transitive.
%----------------------------------------
\begin{solution}
%Uncomment and WRITE YOUR SOLUTION HERE
We can define a relation \(R\) on \(A\) that follows:
\[R = \set{(1, 2), (2, 1), (2, 3), (3, 2)}\]
Here we see that \(R\) is symmetric because it contains \((1,2),(2,1) \in R\) and \((2,3),(3,2) \in R\). It is not reflexive because it doesn't contain the pairs \((1,1),(2,2)(3,3),(4,4) \notin R\)
It is also not transitive because it contains \((1,2),(2,3) \in R\) but not \((1,3) \notin R\).
\end{solution}
%----------------------------------------

\item[(f)] transitive but neither reflexive nor symmetric.
%----------------------------------------
\begin{solution}
We can define a relation \(R\) on \(A\) that follows:
\[R = \set{(1,3),(3,2),(1,2)}\]
We can see that this is transitive because it contains \((1,3)(3,2) \in R\) and we have that \((1,2) \in R\). It is not reflexive because it doesn't contain the pairs \((1,1),(2,2)(3,3),(4,4) \notin R\). It isn't symmetric because it contains \((1,2) \in R\) but not \((2,1) \notin R\).
\end{solution}
%----------------------------------------

\end{itemize}
\end{problem}

\newpage  %Do not delete

\begin{problem}{7.2}
Suppose \(H \subseteq \zz\) is a subset that satisfies \begin{enumerate}
        \item [(a)] If \(x \in H\) then \(-x \in H\)
        \item [(b)] If \(x,y \in H\) then \(x+y \in H\)
    \end{enumerate} Show that the relation \(xRy \iff x-y \in H\) is an equivalence relation. (Hint: first show that \(0 \in H\) )\\
    \textbf{\textit{Unimportant Remark:}} We denote the set of equivalences classes \(\zz/H\) and read it as '\(\zz\) mod H'. This is called the space of cosets in group theory. This problem works much more generally: replace \(\zz\) by any group \(G\), and then \(H\) is called a subgroup of \(G\).
%----------------------------------------
\begin{solution}
%Uncomment and WRITE YOUR SOLUTION HERE
We can first prove that \(0 \in H\) by the first two properties \(a\) and \(b\). By the first one \(a\) if \(x = 0\) then \(-x = -0 = 0 \in H\). By the second property \(b\), since \(x\), \(-x \in H\) then we have that \(x + -x = 0 \in H\).\\
\textbf{Reflexivity}: We want to show that for any \(x \in Z\), \(x-x = 0 \in H\) since \(0 \in H\) as it was hinted. By property \(a\) if \(x \in H\), then \(-x \in H\). Since \(-x\) is just \(x\) with a negative sign then we can conclude that \(xRx\) because \(x - x = 0 \in H\).\\
\textbf{Symmetry}: Let \(x,y \in \mathbb{Z}\) such that \(xRy\) which means that \(x - y \in H\). We want to show that \(y-x \in H\) for symmetry. By the first property \(a\) again if \(x-y \in H\) then \(-(x-y) = y - x \in H\). This implies that if \(xRy\) then \(yRx\).\\
\textbf{Transitivity}: Let \(x,y,z \in \mathbb{Z}\) such that \(xRy\) and \(yRz\). This would mean that \(x-y \in H\) and \(y - z \in H\). We want to show that \(x - z \in H\). Since \(x-y \in H\) and \(y - z \in H\), by property \(b\) we know that \((x-y)+(y-z) \in H\). From this, we simplify like terms and get that \(x-z \in H\). Therefore, if \(xRy\) and \(yRz\), then \(xRz\).
\end{solution}
%----------------------------------------


\end{problem}

\newpage  %Do not delete
%-----------------------------

\begin{problem}{7.3}
Let \(n\zz:= \{nx: x \in \zz\} \subset \zz\) be the subset of all multiples of n. Show that \(n\zz\) satisfies conditions (a) and (b) from the above problem. What is the equivalence relation defined above in this case? What are the space of all cosets?
%----------------------------------------
\begin{solution}
%Uncomment and WRITE YOUR SOLUTION HERE
Let us show that \(n\zz:= \{nx: x \in \zz\} \subset \zz\) satisfies the conditions (a) and (b) from the previous problem.\\
(a). If \(x \in n\zz\) then \(-x\) is also in \(n\zz\). This would be true because if \(x \in n\zz\) then \(x = nk\) for some integer \(k\). Therefore \(-x = -nk = n(-k)\), which is also in the same form.  Thus \(-x\) is in \(n\zz\) which satisfies the condition (a).\\
(b). If \(x \in n\zz\) then \(x = nk\) for some integer \(k\),and if \(y \in n\zz\), then \(y = nl\) for some integer \(l\). Therefore \(x+y = nk+nl = n(k+l)\) where \(k+l\) is also just some integer. Thus \(x+y\) is in \(n\zz\) satisfying the condition (b).\\
For the equivalence relation defined above in this case, it is two integers \(x\) and \(y\) are equivalent if their difference is a multiple of \(n\). They will have the same remainder when divided by \(n\). The space of all cosets would be represented/denoted by \(\zz/n\zz\). It would represent the set of all possible remainders when integers are divided by \(n\).
\end{solution}
%----------------------------------------


\end{problem}

\newpage  %Do not delete
%-----------------------------

\begin{problem}{7.4}
Let $H = \setp{2^m}{m \in \zz}$. A relation $R$ is defined on $\qq_{>0}$, the set of positive rational numbers by: \[a R b \quad \text{if and only if} \quad \frac{a}{b} \in H.\]
\begin{enumerate}
    \item [(a)] Show that $R$ is an equivalence relation
    \item [(b)] Describe the equivalence class $[3]$
    \item [(c)] Prove $[2] = H$.
\end{enumerate}
%----------------------------------------

\begin{solution}
%Uncomment and WRITE YOUR SOLUTION HERE
(a).\\
\textbf{Reflexivity}: For any positive rational number \(a\), we have \(aRa\) if and only if \(\frac{a}{a} = 1 \in H\). Since \(1\) is in \(H\), then \(R\) is reflexive.\\
\textbf{Symmetry}: For any positive rational numbers \(a\) and \(b\), if \(aRb\), then \(\frac{a}{b} \in H\). But if \(\frac{a}{b} \in H\), then \(\frac{b}{a} = \frac{1}{\frac{a}{b}}\) is also in \(H\) since \(H\) is closed under reciprocals. \(\frac{b}{a} = 2^{-m} \in H\) Therefore, \(bRa\), and \(R\) is symmetric.\\
\textbf{Transitivity} For any positive rational numbers \(a,b\) and \(c\), if \(aRb\) and \(bRc\), then \(\frac{a}{b} = 2^m \in H\) and \(\frac{b}{c} = 2^n \in H\). \(H\) would be closed under multiplication so \(\frac{a}{b} * \frac{b}{c} = \frac{a}{c} = 2^{m+n}\in H\). Therefore, \(aRc\), and \(R\) is transitive.\\
(b). The equivalence class for \([3]\) would be:
\begin{align*}
[3] &= \set{a \in \qq_{>0} | aR3}\\
&= \set{a \in \qq_{>0} | \frac{a}{3} = 2^m for some m \in \zz}\\
&= \set{a \in \qq_{>0} | a = 3 * 2^m}
\end{align*}
for some integer \(m\) in \(\zz\). This would consist of all positive rational numbers that when divided by 3 yield a power of 2.\\
(c). We want to prove that every element in \([2]\) is in \(H\) and every element in \(H\) is in \([2]\). Every element in \([2]\) would be in the form of \(\frac{2^n}{1}\) for some non-negative integer \(n\). So it can be expressed as \(2^n\). Since H would be consisting of all powers of \(2\) it would follow that every element in \([2]\) is in H. For every element in H, it is also of the form \(2^m\) for some integer \(m\). We can then consider that \(\frac{2^m}{1}\) is also clearly equivalent to \(2^m\). Every element in H can be expressed as \(\frac{2^m}{1}\) which means it is in \([2]\).
\end{solution}
%----------------------------------------

\end{problem}


\newpage  %Do not delete

\begin{center}
\textbf{Collaborators:}
%List your peers with whom you discussed the Problem Set
\end{center}
\vfill 

\begin{center}
\textbf{References:}
%List any book/website/notes that you used to write your solutions
\end{center}
\begin{itemize}
\item[$\bullet$] [Book(s): Title, Author]
\item[$\bullet$] [Online: \href{http://example.com/}{Link}]
\item[$\bullet$] [Notes: \href{http://example.com/}{Link}]
\end{itemize}

\vfill
\begin{center}
Fin.
\end{center}
\vfill

\end{document}