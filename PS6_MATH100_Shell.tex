\documentclass[11pt]{article}
%------------------------
%Packages
\usepackage[top=0.75in, bottom=1.25in, left=1in, right=1in]{geometry} 
\usepackage{amsmath,amsthm,amssymb} %this is THE math package
\usepackage{mathtools}
\usepackage{tikz}
\usepackage{graphicx}
\usepackage{enumitem}
\usepackage{fancybox}
\usepackage{hyperref}
\usepackage{varwidth}
\usepackage{mdframed}
\usepackage{mathrsfs}
%------------------------
%Fonts I use, uncomment if you like to use them.
%The first is the general font, and the second a math font
\usepackage{mathpazo}
%\usepackage{eulervm}
%------------------------
%This is so that we have standard fonts for the doublestroked symbols
%for reals, naturals etc. regardless of what font you use.
%Don't comment
\AtBeginDocument{
  \DeclareSymbolFont{AMSb}{U}{msb}{m}{n}
  \DeclareSymbolFontAlphabet{\mathbb}{AMSb}}

%----------------------------------------------
%User-defined environments
%Commented because we're not using them in this document
%The only uncommented ones are the Problem and Solution environment

% \newenvironment{theorem}[2][Theorem]{\begin{trivlist}
% \item[\hskip \labelsep {\bfseries #1}\hskip \labelsep {\bfseries #2.}]}{\end{trivlist}}
% \newenvironment{lemma}[2][Lemma]{\begin{trivlist}
% \item[\hskip \labelsep {\bfseries #1}\hskip \labelsep {\bfseries #2.}]}{\end{trivlist}}
% \newenvironment{exercise}[2][Exercise]{\begin{trivlist}
% \item[\hskip \labelsep {\bfseries #1}\hskip \labelsep {\bfseries #2.}]}{\end{trivlist}}
% \newenvironment{question}[2][Question]{\begin{trivlist}
% \item[\hskip \labelsep {\bfseries #1}\hskip \labelsep {\bfseries #2.}]}{\end{trivlist}}
% \newenvironment{corollary}[2][Corollary]{\begin{trivlist}
% \item[\hskip \labelsep {\bfseries #1}\hskip \labelsep {\bfseries #2.}]}{\end{trivlist}}
\newenvironment{problem}[2][Problem\!]{\begin{trivlist}
\item[\hskip \labelsep {\bfseries #1}\hskip \labelsep {\bfseries #2.}]}{\end{trivlist}}
%\newenvironment{sub-problem}[2][]{\begin{trivlist}
%\item[\hskip \labelsep {\bfseries #1}\hskip \labelsep {\bfseries #2}]}{\end{trivlist}}
\newenvironment{solution}{\begin{proof}[\textbf{\textit{Solution}}]}{\end{proof}}
%----------------------------------------------

%----------------------------
%User-defined notations
\newcommand{\zz}{\mathbb Z}   %blackboard bold Z
\newcommand{\qq}{\mathbb Q}   %blackboard bold Q
\newcommand{\ff}{\mathbb F}   %blackboard bold F
\newcommand{\rr}{\mathbb R}   %blackboard bold R
\newcommand{\nn}{\mathbb N}   %blackboard bold N
\newcommand{\cc}{\mathbb C}   %blackboard bold C
\newcommand{\af}{\mathbb A}   %blackboard bold A
\newcommand{\pp}{\mathbb P}   %blackboard bold P
\newcommand{\id}{\operatorname{id}} %for identity map
\newcommand{\im}{\operatorname{im}} %for image of a function
\newcommand{\dom}{\operatorname{dom}} %for domain of a function
\newcommand{\cat}[1]{\mathscr{#1}}   %calligraphic category
\newcommand{\abs}[1]{\left\lvert#1\right\rvert} %for absolute value
\newcommand{\norm}[1]{\left\lVert#1\right\rVert} %for norm
\newcommand{\modar}[1]{\text{ mod }{#1}} %for modular arithmetic
\newcommand{\set}[1]{\left\{#1\right\}} %for set
\newcommand{\setp}[2]{\left\{#1\ \middle|\ #2\right\}} %for set with a property
\newcommand{\card}[1]{\#\,{#1}} %for cardinality of a set

%Re-defined notations
\renewcommand{\epsilon}{\varepsilon}
\renewcommand{\phi}{\varphi}
\renewcommand{\emptyset}{\varnothing}
\renewcommand{\geq}{\geqslant}
\renewcommand{\leq}{\leqslant}
\renewcommand{\Re}{\operatorname{Re}}
\renewcommand{\gcd}{\operatorname{GCD}}
\renewcommand{\Im}{\operatorname{Im}}
%----------------------------

\allowdisplaybreaks
 
\begin{document}
 
\title{Problem Set 6}
\author{[Keene Ho]\\[0.5em]
MATH 100 | Introduction to Proof and Problem Solving | Summer 2022}
\date{} 
\maketitle

%Use \[...\] instead of $$...$$

\begin{problem}{6.1}
Prove that $7 \mid (3^{4n+1} - 5^{2n-1})$ for every positive integer $n$.
\end{problem}
%----------------------------------------
\begin{solution}\hfill \\%Do not delete
We can do proof by induction here for this statement.\\
\textbf{Base Case    \(n = 1\):}\\
When \(n = 1\) we have
\[3^{4(1)+1}-5^{2(1)-1} = 3^{5}-5^1 = 238\]
\(7 | 238\) is true because \(7 * 34 = 238\).\\
\textbf{Inductive Hypothesis:}\\
We can assume that the statement is true for some integer \(k\).
\[7 | 3^{4k+1}-5^{2k-1}\]
We can also rewrite this as 
\[3^{4k+1}-5^{2k-1} = 7l\]
for some integer \(l\).\\
\textbf{Inductive Step}:\\
We now need to prove this statement is true for \(k + 1\).
\[7 | 3^{4(k+1)+1}-5^{2(k+1)-1}\]
We can then rewrite this equation.
\begin{align*}
3^{4(k+1)+1}-5^{2(k+1)-1} &= 3^{4k+4+1}-5^{2k+2-1}\\
&= 3^{4k+1}*3^{4}-5^{2k-1}*5^{2}\\
&= 3^{4k+1}*81 - 5^{2k-1}*25\\
&= (7l+5^{2k-1})81 - 5^{2k-1}*25\\
&= 5^{2k-1}(81-25) +7l * 81\\
&= 5^{2k-1}(56) +7l * 81\\
&= 7(5^{2k-1}*8+81l)
\end{align*}
As we can see this is a factor/multiple of \(7\). Therefore \(7 | 3^{4(k+1)+1}-5^{2(k+1)-1}\) is true. This completes the proof and we have shown that $7 \mid (3^{4n+1} - 5^{2n-1})$ for every positive integer $n$.
\end{solution}
%----------------------------------------

\newpage %Do not delete

\begin{problem}{6.2}
A sequence $\set{a_n}$ is defined recursively by $a_1 = 1,\, a_2 = 4,\, a_3 = 9$ and
\[a_n = a_{n-1} - a_{n-2} + a_{n-3} + 2(2n - 3),\quad \text{for $n \geq 4$}\]
\begin{itemize}[itemsep=3em]
\item[(a)] Use the recursive relation to compute $a_4,\,a_5,\,a_6,\,a_7$.
%----------------------------------------
\begin{solution}\hfill %Do not delete
%Uncomment and WRITE YOUR SOLUTION HERE
\begin{align*}
    a_4 &= a_3-a_2+a_1+2(2(4)-3)\\
    &= 9-4+1+2(8-3)\\
    &= 9-4+1+2(5)\\
    &= 9-4+1+10\\
    &= 16
\end{align*}
\begin{align*}
    a_5 &= a_4-a_3+a_2+2(2(5)-3)\\
    &= 16-9+4+2(10-3)\\
    &= 16-9+4+2(7)\\
    &= 16-9+4+14\\
    &= 25
\end{align*}
\begin{align*}
    a_6 &= a_5-a_4+a_3+2(2(6)-3)\\
    &= 25-16+9+2(12-3)\\
    &= 25-16+9+2(9)\\
    &= 25-16+9+18\\
    &= 36
\end{align*}
\begin{align*}
    a_7 &= a_6-a_5+a_4+2(2(7)-3)\\
    &= 36-25+16+2(14-3)\\
    &= 36-25+16+2(11)\\
    &= 36-25+16+22\\
    &= 49
\end{align*}
\end{solution}
%----------------------------------------

\item[(b)] Looking at the values for $a_1,\,a_2,\,a_3,\,a_4,\,a_5,\,a_6$ and $a_7$, conjecture a formula for $a_n$, that is, an expression in terms of $n$.
%----------------------------------------
\begin{solution}\hfill %Do not delete
\begin{align*}
a_1 &= 1 = 1^2 \\
a_2 &= 4  = 2^2\\
a_3 &= 9  = 3^2\\
a_4 &= 16  = 4^2\\
a_5 &= 25 = 5^2\\
a_6 &= 36 = 6^2\\
a_7 &= 49 = 7^2\\
\end{align*}
As we can see each \(a_n\) seems to be a perfect square. We can check if this pattern holds for \(a_8 = 64\).
\begin{align*}
a_8 &= a_7 - a_6 + a_5 + 2(2(8) - 3) \\
&= 49 - 36 + 25 + 2(16 - 3) \\
&= 49 - 36 + 25 + 2(13) \\
&= 49 - 36 + 25 + 26 \\
&= 64
\end{align*}
Therefore \(a_n = n^2\)
\end{solution}
%----------------------------------------

\item[(c)] Prove your conjecture using an appropriate principle of mathematical induction, using the recursive relation.
%----------------------------------------
\begin{solution}\hfill %Do not delete
%Uncomment and WRITE YOUR SOLUTION HERE
\\
\textbf{Base Case \(n = 1\)}\\
When \(n = 1\) we have 
\[a_1 = 1^2 = 1\]
\textbf{Inductive Hypothesis}\\
We can assume that \(a_k = k^2\) for some positive integer \(k\).\\
\textbf{Inductive Step}\\
We want to show that \(a_{k+1} = (k+1)^2\). Using the given recursive relation:
\[a_{k+1} = a_k - a_{k-1} + a_{k-2} + 2(2(k+1)-3)\]
Now we can substitute in our inductive hypothesis \(a_k = k^2\) and that \(a_{k-1} = (k-1)^2\).
\begin{align*}
a_{k+1} &= k^2-(k-1)^2+a_{k-2}+2(2(k+1)-3)\\
&=k^2-(k^2-2k+1)+a_{k-2}+2(2k+2-3)\\
&= k^2-k^2+2k-1+a_{k-2}+2(2k-1)\\
&= 2k - 1 + a_{k-2}+4k-2\\
&= 6k+a_{k-2}-3
\end{align*}
By our hypothesis we can conclude that \(a_{k-2} = (k-2)^2\).
\begin{align*}
    a_{k+1} &= 6k + (k-2)^2 - 3\\
    &= 6k + (k^2-4k+4)-3\\
    &= k^2 + 2k + 1\\
    &= (k+1)^2
\end{align*}
This completes the induction step as we have shown that \(a_{k+1} = (k+1)^2\). We have shown that \(a_n = n^2\) for all positive integers \(n\).
\end{solution}
%----------------------------------------

\end{itemize}
\end{problem}

\newpage  %Do not delete

\begin{center}
\textbf{Collaborators:}
%List your peers with whom you discussed the Problem Set
\end{center}
\vfill 

\begin{center}
\textbf{References:}
%List any book/website/notes that you used to write your solutions
\end{center}
\begin{itemize}
\item[$\bullet$] [Book(s): Title, Author]
\item[$\bullet$] [Online: \href{http://example.com/}{Link}]
\item[$\bullet$] [Notes: \href{http://example.com/}{Link}]
\end{itemize}

\vfill
\begin{center}
Fin.
\end{center}
\vfill

\end{document}