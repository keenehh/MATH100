\documentclass[11pt]{article}
%------------------------
%Packages
\usepackage[top=0.75in, bottom=1.25in, left=1in, right=1in]{geometry} 
\usepackage{amsmath,amsthm,amssymb} %this is THE math package
\usepackage{mathtools}
\usepackage{tikz}
\usepackage{graphicx}
\usepackage{enumitem}
\usepackage{fancybox}
\usepackage{hyperref}
\usepackage{varwidth}
\usepackage{mdframed}
\usepackage{mathrsfs}
%------------------------
%Fonts I use, uncomment if you like to use them.
%The first is the general font, and the second a math font
\usepackage{mathpazo}
%\usepackage{eulervm}
%------------------------
%This is so that we have standard fonts for the doublestroked symbols
%for reals, naturals etc. regardless of what font you use.
%Don't comment
\AtBeginDocument{
  \DeclareSymbolFont{AMSb}{U}{msb}{m}{n}
  \DeclareSymbolFontAlphabet{\mathbb}{AMSb}}

%----------------------------------------------
%User-defined environments
%Commented because we're not using them in this document
%The only uncommented ones are the Problem and Solution environment

% \newenvironment{theorem}[2][Theorem]{\begin{trivlist}
% \item[\hskip \labelsep {\bfseries #1}\hskip \labelsep {\bfseries #2.}]}{\end{trivlist}}
% \newenvironment{lemma}[2][Lemma]{\begin{trivlist}
% \item[\hskip \labelsep {\bfseries #1}\hskip \labelsep {\bfseries #2.}]}{\end{trivlist}}
% \newenvironment{exercise}[2][Exercise]{\begin{trivlist}
% \item[\hskip \labelsep {\bfseries #1}\hskip \labelsep {\bfseries #2.}]}{\end{trivlist}}
% \newenvironment{question}[2][Question]{\begin{trivlist}
% \item[\hskip \labelsep {\bfseries #1}\hskip \labelsep {\bfseries #2.}]}{\end{trivlist}}
% \newenvironment{corollary}[2][Corollary]{\begin{trivlist}
% \item[\hskip \labelsep {\bfseries #1}\hskip \labelsep {\bfseries #2.}]}{\end{trivlist}}
\newenvironment{problem}[2][Problem\!]{\begin{trivlist}
\item[\hskip \labelsep {\bfseries #1}\hskip \labelsep {\bfseries #2.}]}{\end{trivlist}}
%\newenvironment{sub-problem}[2][]{\begin{trivlist}
%\item[\hskip \labelsep {\bfseries #1}\hskip \labelsep {\bfseries #2}]}{\end{trivlist}}
\newenvironment{solution}{\begin{proof}[\textbf{\textit{Solution}}]}{\end{proof}}
%----------------------------------------------

%----------------------------
%User-defined notations
\newcommand{\zz}{\mathbb Z}   %blackboard bold Z
\newcommand{\qq}{\mathbb Q}   %blackboard bold Q
\newcommand{\ff}{\mathbb F}   %blackboard bold F
\newcommand{\rr}{\mathbb R}   %blackboard bold R
\newcommand{\nn}{\mathbb N}   %blackboard bold N
\newcommand{\cc}{\mathbb C}   %blackboard bold C
\newcommand{\af}{\mathbb A}   %blackboard bold A
\newcommand{\pp}{\mathbb P}   %blackboard bold P
\newcommand{\id}{\operatorname{id}} %for identity map
\newcommand{\im}{\operatorname{im}} %for image of a function
\newcommand{\dom}{\operatorname{dom}} %for domain of a function
\newcommand{\cat}[1]{\mathscr{#1}}   %calligraphic category
\newcommand{\abs}[1]{\left\lvert#1\right\rvert} %for absolute value
\newcommand{\norm}[1]{\left\lVert#1\right\rVert} %for norm
\newcommand{\modar}[1]{\text{ mod }{#1}} %for modular arithmetic
\newcommand{\set}[1]{\left\{#1\right\}} %for set
\newcommand{\setp}[2]{\left\{#1\ \middle|\ #2\right\}} %for set with a property
\newcommand{\card}[1]{\#\,{#1}} %for cardinality of a set

%Re-defined notations
\renewcommand{\epsilon}{\varepsilon}
\renewcommand{\phi}{\varphi}
\renewcommand{\emptyset}{\varnothing}
\renewcommand{\geq}{\geqslant}
\renewcommand{\leq}{\leqslant}
\renewcommand{\Re}{\operatorname{Re}}
\renewcommand{\gcd}{\operatorname{GCD}}
\renewcommand{\Im}{\operatorname{Im}}
%----------------------------

\allowdisplaybreaks
 
\begin{document}
 
\title{Problem Set 3}
\author{[Keene Ho]\\[0.5em]
MATH 100 | Introduction to Proof and Problem Solving | Summer 2023}
\date{} 
\maketitle

%Use \[...\] instead of $$...$$

\begin{problem}{3.1}
For a triangle $T$, let $r(T)$ denote the ratio of the length of the longest side of $T$ to the length of the smallest side of $T$. Let $\triangle$ denote the set of all triangles and let
\[P(T_1,\ T_2): r(T_2) \geq r(T_1).\]
be an open sentence where the domain of both $T_1$ and $T_2$ is $\triangle$.
\begin{itemize}[itemsep=3em]
\item[(a)] Express the following quantified statement in words:
\begin{equation}\label{eq1}
\exists T_1 \in \triangle,\ \forall T_2 \in \triangle,\ P(T_1, T_2).\tag{$*$}
\end{equation}
%----------------------------------------
\begin{solution}
There exists a triangle \(T_1\) in the set of all triangles, such that for all triangles \(T_2\) in the set of all triangles the ratio of the length of the longest side of \(T_2\) to the length of the smallest side of \(T_2\) is greater than or equal to the ratio of the length of the longest side of \(T_1\) to the length of the smallest side of \(T_1\).
\end{solution}
%----------------------------------------

\item[(b)] Express the negation of the quantified statement in (\ref{eq1}) in symbols.
%----------------------------------------
\begin{solution}
\(\forall T_1 \in \triangle,\ \exists T_2 \in \triangle,\ \neg P(T_1, T_2)\). 
\end{solution}
%----------------------------------------

\item[(c)] Express the negation of the quantified statement in (\ref{eq1}) in words.
%----------------------------------------
\begin{solution}
For every triangle \(T_1\) in the set of all triangles, there exists a triangle \(T_2\) in the set of all triangles such that the ratio of the length of the longest side of \(T_2\) to the length of the smallest side of \(T_2\) is not greater than the ratio of the length of the longest side of \(T_1\) to the length of the smallest side of \(T_1\).
\end{solution}
%----------------------------------------

\end{itemize}
\end{problem}

\newpage %Do not delete

\begin{problem}{3.2}\hfill
\begin{itemize}[itemsep=3em]
\item[(a)] Let $x \in \rr$. Prove that if $0 < x < 1$, then $x^2 - 2x + 2 \neq 0$.
%----------------------------------------
\begin{solution}
We can first complete the square.
\[x^2-2x+2 = 0\]
\[x^2-2x+1+1 = 0\]
\[(x-1)^2 + 1 = 0\]
When \(0 < x < 1\), \(x - 1\) will be negative but the \((x-1)^2\) will be positive since squaring a negative number will result in a positive number. Adding \(+1\) will give a value greater than \(1\) as well. So \((x+1)^2 + 1\) is always greater than 1. This implies that for all values of \(x\) in the range \(0 < x < 1\), then \(x^2-2x+2 \neq 0\).
\end{solution}
%----------------------------------------

\item[(b)] For two sets $A$ and $B$, prove $B \subseteq A$ is the same as proving the implication 
\[x \in B \implies x \in A\]
Prove that for any set $X$, we have $\emptyset \subseteq X$.
%----------------------------------------
\begin{solution}
%Uncomment and WRITE YOUR SOLUTION HERE
If \(B \subseteq A\), then any element \(x\) in set \(B\) must also be in set \(A\) which can be rewritten as \(x \in B \implies x \in A\). If \(x \in B \implies x \in A\) holds true for all \(x\), it means that every element in set \(B\) is also in set \(A\), which would also imply that \(B \subseteq A\). So for any set \(X\) the empty set \(\emptyset\) is a subset of every set, including \(X\). WTS that every element of \(\emptyset\) is also an element of \(X\) but since there are no elements in \(\emptyset\) this is vacuously true.
\[x \in \emptyset \implies x \in X\]
\(x \in \emptyset\) is always going to be false because there are no elements in the empty set.
\end{solution}
%----------------------------------------

\end{itemize}
\end{problem}

\newpage  %Do not delete

\begin{problem}{3.3}\hfill
\begin{itemize}[itemsep=3em]
\item[(a)] Prove that if $a$ and $c$ are odd integers, then $ab + bc$ is even for every integer $b$.
%----------------------------------------
\begin{solution}
%Uncomment and WRITE YOUR SOLUTION HERE
    If \(a\) and \(c\) are odd integers, then \(ab+bc\) is even for every integer \(b\). Let \(a\) and \(c\) be odd integers.  By definition an odd integer it can be written as \(a = 2k + 1\) and \(c = 2l + 1\) where \(k\) and \(l\) are integers.
    \[ab+bc = (2k+1)b+b(2l+1)\]
    \[=2kb+b+2lb+b\]
    \[=2kb+2b+2lb\]
    \[=2(kb+b+lb)\]
    \(kb+b+lb \in \zz\) so \(ab+bc\) is even for every integer \(b\).
\end{solution}
%----------------------------------------

\item[(b)] Let $x \in \zz$. Prove that if $2^{2x}$ is an odd integer, then $2^{-2x}$ is an odd integer.
%----------------------------------------
\begin{solution}
The only way that \(2^{2x}\) is an odd integer is if \(x = 0\). \\
Case 1: x is a positive integer\\
If \(x\) is a positive integer, then \(2x\) is an even integer as its a multiple of 2 (plus by definition) and therefore \(2^{2x}\) is also an even integer. 2 to the power of an integer is just \(2 * 2 * 2...\) which is just even.\\
Case 2: x is a negative integer\\
If we assume that \(x\) is a negative integer then \(-2x\) would still be an even integer and therefore \(2^{-2x}\) is still an even integer. So for \(2^{2x}\) to be an odd integer, \(x\) must be equal to 0. 
\[2^{2*0} = 2^{0} = 1\]
With \(x = 0\), then \(2^{-2x} = 2^{0} = 1\) making it an odd integer. This proves that with \(2^{2x}\) is an odd integer, then \(2^{-2x}\) is also an odd integer.
\end{solution}
%----------------------------------------

\end{itemize}
\end{problem}

\newpage  %Do not delete

\begin{center}
\textbf{Collaborators:}
%List your peers with whom you discussed the Problem Set
\end{center}
\vfill 

\begin{center}
\textbf{References:}
%List any book/website/notes that you used to write your solutions
\end{center}
\begin{itemize}
\item[$\bullet$] [Book(s): Title, Author]
\item[$\bullet$] [Online: \href{http://example.com/}{Link}]
\item[$\bullet$] [Notes: Mostly Modules. Also referenced some stuff from MATH110]
\end{itemize}

\vfill
\begin{center}
Fin.
\end{center}
\vfill

\end{document}